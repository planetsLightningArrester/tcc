\documentclass[11pt]{abntex2}
\usepackage[utf8]{inputenc}
\usepackage[brazil]{babel}
\usepackage{indentfirst}
\usepackage{graphicx}
\usepackage{float}
\usepackage{enumitem}
\usepackage{mathtools}
\usepackage{pdflscape}
\usepackage{textcomp}
\usepackage[num, overcite]{abntex2cite}
\citebrackets[]

\titulo{Módulo remoto para ensaios de vibração}
\autor{Francisco Gomes Soares Sanches Manso}
\data{\today}
\instituicao{%
Universidade Federal de Minas Gerais - UFMG
\par
Escola de Engenharia
\par
Kotchergenko Engenharia Ltda.}
\local{Belo Horizonte}
\preambulo{Monografia apresentada durante o Seminário dos Trabalhos de Conclusão do Curso de Graduação em
Engenharia Elétrica da UFMG, como parte dos requisitos necessários à obtenção do título de Engenheiro Eletricista}
\orientador[Orientador:]{Ricardo de Oliveira Duarte}
\coorientador[Supervisor:]{Bruno Freitas Brant}

\begin{document}
\makeatletter
	\imprimircapa
	\imprimirfolhaderosto

	\begin{resumo}
		
	\end{resumo}

	\tableofcontents
	\newpage
	\listoffigures
	\newpage
	
	\chapter{Introdução}
		A mineração no Brasil possui grande importância na economia atual do país e
		do mundo e é um dos setores em maior expansão. Atividades nessa área já
		representam em torno de 5\% do PIB do país e geram mais de dois milhões de
		empregos diretos e indiretos.\cite{pib}

		Novas tecnologias vêm alavancando esse setor, buscando aumentar a eficiência
		de produção e transporte e o aproveitamento de resíduos para a
		transformação em insumos. A Escola de Engenharia da Universidade Federal de
		Minas Gerais (UFMG), por exemplo, desenvolve metodologias de calcificação
		dos resíduos da mineração, os tornando matéria-prima para a fabricação de
		produtos das áreas de construção civil. Esse reaproveitamento chega a
		proporcionar uma redução de até 40\% no custo das obras.\cite{mineracaoUFMG}
		
		O setor de mineração conta com diversas estruturas de grande porte em
		terminais portuários e ferrovias por todo o Brasil. A manutenção
		preditiva e o diagnóstico de falha são duas atividades de extrema
		importância no âmbito de possibilitar a segurança dos operadores, a
		redução de custos em bloqueios de produção por falhas e uma melhor
		modelagem da dinâmicas das estruturas utilizadas. Nesse sentido,
		diversas empresas da área baseiam suas atividades em três grandes
		pilares: a metodologia teórica de análise de estruturas, a capacidade de
		modelagem e simulação via \textit{software} e um preciso e confiável
		ensaio de campo para a obtenção de dados.
		
		Ensaios de campo de vibração e extensometria são comumente realizados
		utilizando equipamentos capazes de fazer aquisição de dados em tempo
		real de vários canais simultaneamente. Os ensaios de vibração, por
		exemplo, utilizam sensores piezoelétricos uniaxiais que são ligados em
		sistemas de aquisição, como o NI-9234 da National
		Instruments\textsuperscript{TM}.
		
		Os dados de vibração são obtidos por meio de sensores piezoelétricos com
		eletrônica integrada, conhecidos como sensores IEPE ou \textit{Integrated
		Electronics Piezo-Electric}. Materiais piezoelétricos são cristais capazes de
		gerar uma tensão elétrica após os aplicar uma força mecânica.
		Os transdutores IEPE pré-amplificam esse sinal de forma a possibilitar a
		condução dos mesmos através de cabos coaxiais.

		Tais ensaios são realizados em peneiras vibratórias de mineração,
        transportadores de correia e outras máquinas de áreas portuárias e ferroviárias.

        \bibliography{./bibliografia}

\end{document}