\documentclass[11pt]{abntex2}
\usepackage[utf8]{inputenc}
\usepackage[brazil]{babel}
\usepackage{indentfirst}
\usepackage{graphicx}
\usepackage{float}
\usepackage{enumitem}
\usepackage{mathtools}
\usepackage{pdflscape}
\usepackage{textcomp}
\usepackage[num, overcite]{abntex2cite}
\citebrackets[]
    
\titulo{Módulo remoto para ensaios de vibração}
\autor{Francisco Gomes Soares Sanches Manso}
\data{\today}
\instituicao{%
Universidade Federal de Minas Gerais - UFMG
\par
Escola de Engenharia
\par
Kotchergenko Engenharia Ltda.}
\local{Belo Horizonte}
\preambulo{Monografia apresentada durante o Seminário dos Trabalhos de Conclusão do Curso de Graduação em
Engenharia Elétrica da UFMG, como parte dos requisitos necessários à obtenção do título de Engenheiro Eletricista}
\orientador[Orientador:]{Ricardo de Oliveira Duarte}
\coorientador[Supervisor:]{Bruno Freitas Brant}

\begin{document}
\makeatletter
	\imprimircapa
	\imprimirfolhaderosto

	\begin{resumo}
		
	\end{resumo}

	\tableofcontents
	\newpage
	\listoffigures
	\newpage
	
	\chapter{Introdução}
		A mineração no Brasil possui grande importância na economia atual do país e
		do mundo e é um dos setores em maior expansão. Atividades nessa área já
		representam em torno de 5\% do PIB do país e geram mais de dois milhões de
		empregos diretos e indiretos.\cite{pib}

		Novas tecnologias vêm alavancando esse setor, buscando aumentar a eficiência
		de produção e transporte e o aproveitamento de resíduos para a
		transformação em insumos. A Escola de Engenharia da Universidade Federal de
		Minas Gerais (UFMG), por exemplo, desenvolve metodologias de calcificação
		dos resíduos da mineração, os tornando matéria-prima para a fabricação de
		produtos das áreas de construção civil. Esse reaproveitamento chega a
		proporcionar uma redução de até 40\% no custo das obras.\cite{mineracaoUFMG}
		
		O setor de mineração conta com diversas estruturas de grande porte em
		terminais portuários e ferrovias por todo o Brasil. A manutenção
		preditiva e o diagnóstico de falha são duas atividades de extrema
		importância no âmbito de possibilitar a segurança dos operadores, a
		redução de custos em bloqueios de produção por falhas e uma melhor
		modelagem da dinâmicas das estruturas utilizadas. Nesse sentido,
		diversas empresas da área baseiam suas atividades em três grandes
		pilares: a metodologia teórica de análise de estruturas, a capacidade de
		modelagem e simulação via \textit{software} e um preciso e confiável
		ensaio de campo para a obtenção de dados.
		
		Ensaios de campo de vibração e extensometria são comumente realizados
		utilizando equipamentos capazes de fazer aquisição de dados em tempo
		real de vários canais simultaneamente. Os ensaios de vibração, por
		exemplo, utilizam sensores piezoelétricos uniaxiais que são ligados em
		sistemas de aquisição, como o NI-9234 da National
		Instruments\textsuperscript{TM}.
		
		Os dados de vibração são obtidos por meio de sensores piezoelétricos com
		eletrônica integrada, conhecidos como sensores IEPE ou \textit{Integrated
		Electronics Piezo-Electric}. Materiais piezoelétricos são cristais capazes de
		gerar uma tensão elétrica após os aplicar uma força mecânica.
		Os transdutores IEPE pré-amplificam esse sinal de forma a possibilitar a
		condução dos mesmos através de cabos coaxiais.

		Tais ensaios são realizados em peneiras vibratórias de mineração,
        transportadores de correia e outras máquinas de áreas portuárias e ferroviárias.

		\chapter{Revisão Bibliográfica}
			No processo de amostragem de um sinal qualquer, deve-se filtrar o sinal com
			um filtro anti-\textit{aliasing} e aplicar os ganhos necessários para adequar
			o sinal à faixa de leitura do conversor analógico-digital. Os processos são
			ilustrados abaixo.

			\begin{figure}[H]
				\centering
				\includegraphics[width=\linewidth]{../../Fotos/Diagramas/sinalFaaADC/sinalFaaADC.png}
				\caption{Condicionamento do sinal analógico}
				\label{fig:condicionamentoSinal}
			\end{figure}

			A seguir, serão descritos cada parte que compõe este processo.

			\section{Transdutores piezoelétricos}
				Transdutores piezoelétricos são largamente utilizados para
				monitoramento industrial. Estes transdutores possuem uma eletrônica
				integrada que pré-amplifica e condiciona o sinal para melhor
				desempenho da medição. Os transdutores piezoelétricos com eletrônica
				integrada são denominados IEPE (\textit{Integrated Electronics
				Piezo-Electric}). Entre os transdutores, ou medidores, do tipo IEPE
				mais comuns, encontram-se os transdutores de pressão e de
				aceleração.

				Os medidores IEPE possuem variadas fichas técnicas, com diferentes
				faixas de alimentação, faixas de sinal de saída e tipos de sinal de
				saída. No caso dos medidores de aceleração com saída em tensão, os
				terminais de alimentação e de sinal de saída são compartilhados.

				Os transdutores de aceleração necessitam ser alimentados por uma
				tensão entre $18V$ e $30V$ com uma corrente constante de polarização
				entre $2ma$ e $10ma$, podendo variar de medidor para medidor. Para
				isso, utiliza-se uma fonte de tensão em série com uma fonte de
				corrente. Assim, o sinal de saída é a tensão imediatamente após a
				fonte de corrente. O esquemático abaixo ilustra o circuito elétrico
				equivalente da alimentação de um transdutor IEPE com saída em
				tensão.

				A conversão do sinal de saída para a grandeza de interesse é dada
				por uma relação linear de tensão e da unidade da grandeza. Essa
				relação é denominada sensibilidade. Como exemplo, toma-se o
				transdutor IEPE AC102, que possui sensibilidade de $100mV/g$.
				Assim, com um valor de fundo de escala de $\pm50g$, têm-se variações
				no valor da tensão de saída de $\pm5V$.\cite{ctc}
				\newpage

				\begin{figure}[H]
					\centering
					\includegraphics[width=\linewidth]{../../Fotos/Diagramas/iepe.png}
					\caption{Diagrama de interligação de um transdutor \textit{IEPE}}
					\label{fig:ligacaoIEPE}
				\end{figure}

			\section{O filtro anti-\textit{aliasing} e o teorema da amostragem}
				Seja um sinal $S(t)$ com maior componente de frequência
				$f_{max}$. Segundo o Teorema da Amostragem, ou Teorema de
				\textit{Nyquist}, esse sinal deve ser amostrado com uma
				frequência $f_s$ tal que $f_s>2f_{max}$. Caso contrário, o
				espectro de frequência irá sobrepor-se, impossibilitando a
				reconstrução correta do sinal no tempo. Esse fenômeno é denominado
				de \textit{aliasing} e a frequência $2\*f_{max}$ é definida como
				frequência de \textit{Nyquist}. \cite{nyquist}

				Como exemplo, assume-se o sinal $S(t) = 0,7\*sen(2\*\pi\*50\*t)
				+ sen(2\*\pi\*120\*t)$. O sinal é amostrado com uma frequência
				$f_s = 250Hz$. Os gráficos do sinal original, do sinal
				amostrado e da FFT (\textit{Fast Fourier Transform}), gerada a
				partir da aquisição do sinal na dada frequência de amostragem,
				são exibidos na Figura ~\ref{fig:exemploAliasing250} e foram gerados
				através do \textit{software} Scilab

				Como a maior componente de frequência do sinal $S(t)$ é $120Hz$,
				a frequência de \textit{Nyquist} é $240Hz$. Assim, a taxa de
				amostragem satisfaz o Teorema da Amostragem. Com isso possibilita-se
				uma correta visualização da FFT e da reconstrução do sinal no tempo.

				Ao fazer uma segunda aquisição do mesmo sinal com uma taxa de
				amostragem $f_s = 200Hz$, obtém-se os gráficos exibidos na
				Figura ~\ref{fig:exemploAliasing200}. Como a frequência de amostragem
				é menor que a frequência de \textit{Nyquist} para este sinal, ocorre
				\textit{aliasing} e o sinal não é reconstruído corretamente.
				Esse erro também pode ser visto pela FFT, que apresenta
				frequências diferentes do sinal original.

				\newpage

				\begin{figure}[!ht]
					\centering
					\includegraphics[width=\linewidth]{../../Fotos/aliasingFs250.png}
					\caption{Exemplo de sinal sem \textit{aliasing} amostrado à $250Hz$}
					\label{fig:exemploAliasing250}
				\end{figure}

				\begin{figure}[H]
					\centering
					\includegraphics[width=\linewidth]{../../Fotos/aliasingFs200.png}
					\caption{Exemplo de sinal com \textit{aliasing} amostrado à $200Hz$}
					\label{fig:exemploAliasing200}
				\end{figure}

				\pagebreak

				O filtro anti-\textit{aliasing} trata-se de um filtro passa-faixa
				ou passa-baixa. Como no projeto não há restrições com relação à
				baixas frequências, o filtro anti-\textit{aliasing} deste projeto
				é um filtro passa-baixa. Um filtro passa-baixa ideal possui
				característica de módulo em frequência mostrada na Figura ~\ref{fig:filtroIdeal}.

				\begin{figure}[!ht]
					\centering
					\begin{minipage}{.4\linewidth}
						\centering
						\includegraphics[width=.8\linewidth]{../../Fotos/Diagramas/sinalFaaADC/filtroIdeal.png}
						\caption{Resposta em frequência de um filtro ideal}
						\label{fig:filtroIdeal}
					\end{minipage}
					\hfill\vline\hfill
					\begin{minipage}{.4\linewidth}
						\centering
						\includegraphics[width=\linewidth]{../../Fotos/Diagramas/sinalFaaADC/filtroReal.png}
						\caption{Resposta em frequência de um filtro real}
					\end{minipage}
				\end{figure}

				Entretanto, os filtros passa-baixa reais possuem uma banda a
				mais: a banda de transição. Essa banda representa os sinais de
				frequência cujas amplitudes não foram atenuadas suficientemente,
				de forma que não possam ser lidas pelo Conversor Analógico-Digital
				(ADC). Assim, se forem desprezadas na admissão da taxa de amostragem,
				podem gerar \textit{aliasing}.

				Com isso, em uma primeira análise, pode-se dizer que a frequência de
				amostragem deve satisfazer $f_s>2*f_{sb}$, em que $f_{sb}$ é chamada
				\textit{stop band frequency}, ou frequência da banda de rejeição.

				Não existe método capaz de reverter o \textit{aliasing} após o
				sinal ser amostrado. Então, o que deve ser feito é assumir uma
				faixa de frequência de interesse e utilizar circuitos capazes de
				atenuar frequências acima da frequência de \textit{Nyquist}.
				Essa atenuação deve ser tal que o ADC não possua resolução capaz
				de detectar as componentes de alta frequência. Os circuitos
				responsáveis por garantir que não ocorra \textit{aliasing} são
				denominados filtro anti-\textit{aliasing} (FAA).

			\section{Conversor analógico-digital}
				O conversor analógico-digital, ou ADC, é responsável por digitalizar
				um sinal analógico. A conversão é referenciada às tensões de alimentação
				do ADC e graduada de acordo com o número de \textit{bits}. Como exemplo,
				um conversor alimentado por uma tensão de $5V$ e com $10$ \textit{bits} de
				resolução possui $2^{10}$ divisões. Assim, a menor divisão do conversor é
				de $5/2^{10} = 4,88mV$ e, consequentemente, este é o menor valor que pode
				ser detectado pelo ADC. Se a resolução fosse de $12$ \textit{bits} ao invés
				de $10$ \textit{bits}, a resolução do conversor seria de $5/2^{12}=1,22mV$.
				Este também é o passo do conversor, também definido como LSB
				(\textit{Least Significant Bit}), ou seja, todas os valores digitalizados
				são múltiplos da resolução.

		\chapter{Metodologia}

			\section{Levantamento de requisitos}
				Abaixo são descritos os principais requisitos do projeto.

				\begin{itemize}
					\item O sistema deve ser capaz de amostrar dados de vibração da
					estrutura e enviar remotamente para um computador com uma outra
					placa, que será brevemente descrita neste relatório. O alcance
					deve ser superior à 20 metros.
					
					\item Possuir quatro canais com entradas para conectores BNC.
					
					\item Ser capaz de aquisitar sinais de 1kHz com uma resolução de $50mg$, sendo g
					a aceleração da gravidade.
					
					\item Ser capaz de operar por 2h com um \textit{powerbank} modelo CB078 com saída de $5V$
					e capacidade de $2200mAh$.
					
					\item Todos os componentes, com exceção da fabricação das
					PCBs(\textit{Printed Circuit Board}), devem ser encontrados à
					venda no Brasil.
				\end{itemize}

			\section{Análise de requisitos}
			
				A partir dos requisitos, a placa deve possuir, essencialmente, uma
				ou mais fontes reguláveis, um ou mais conversores ADC
				(\textit{analog-digital converter}), unidades de lógica programável
				e sistemas de telecomunicações.

				Considerando uma eficiência de $80\%$ do conversor DC/DC interno do
				\textit{powerbank}, a capacidade total do modelo CB078 é admitida como sendo
				1760mAh. Assim, para operar durante 2h o sistema deve ter consumo de
				corrente inferior a 880mA e possuir tensão de alimentação de $5V$.
				
				Os sensores de vibração utilizados na empresa são do modelo CTC
				AC102. Estes operam quando aplicado uma corrente de excitação fixa,
				além da alimentação. Tal sensor irá basear o projeto do circuito de
				condicionamento do sinal.
				
				As principais características do transdutor AC102 são apresentadas
				abaixo.\cite{ctc}
				
				\begin{itemize}
					\item Sensibilidade: $100mV/g$
					\item Tensão de alimentação: $18 - 30 VDC$
					\item Faixa de operação: $\pm 50g$
					\item Corrente de excitação: $2 - 10mA$
					\item Tensão de saída polarizada: $10 - 14 VDC$
					\item Faixa de passagem: $1 - 10.000 Hz$
				\end{itemize}
				
				A última consideração a ser feita é com relação aos componentes
				possuírem disponibilidade no Brasil. Infelizmente, existe uma grande
				dificuldade de se encontrar componentes com características
				desejáveis para projetos que exigem precisão. Assim, a ordem de
				definição da arquitetura do projeto pode ser invertida. Por exemplo,
				no Brasil existe uma variedade maior de fontes de alimentação do que
				de microcontroladores. Dessa forma, os componentes da fonte de
				alimentação serão determinados após a escolha do microcontrolador.
				
				Todos os valores acima citados definem as condições de contorno do
				projeto a ser desenvolvido.

			\section{Condicionamento de sinais}

				A alimentação do sensor deve ser de $18V$ a $30V$. Como a entrada de
				alimentação do sistema é de $5V$, optou-se por alimentar o sensor
				piezoelétrico com $19V$ utilizando uma fonte chaveada \textit{boost}, que será
				descrita mais à frente.

				A fonte de corrente deve possuir valor de corrente de $2mA$ a $10mA$. Um valor muito
				utilizado comercialmente é $4mA$, por ter um consumo intermediário e não trabalhar
				muito próximo dos valores máximos e mínimos absolutos.

				A fim de desenvolvimento do projeto, será considerado que a tensão
				média de saída será de $14V$. Mais à frente irá ser demonstrado que
				essa premissa é indiferente. Esta será a tensão média em que o sinal
				de $\pm 5V$ irá varia, excursionando de $7V$ a $17V$. Assim, como o
				sinal excursiona sobre uma tensão positiva, não há a necessidade de
				utilização de fontes simétricas. Dessa forma, o filtro
				anti-\textit{aliasing} deve possuir amplificadores operacionais
				capazes de operar com alimentação não-simétrica de 19V.

			\section{Dimensionamento do ADC}
				
				As condições de contorno do projeto impõem uma resolução de $50mg$. Como
				visto na análise de requisitos e no diagrama da Figura ~\ref{fig:condicionamentoSinal},
				têm-se os seguintes dados.
				\begin{itemize}
					
					\item Tensão média de entrada: $V_{i(av)} = 14V$;
					\item Excursão total do sinal ($\pm5V$): $V_{i(pp)} = 10V$.
					
				\end{itemize}
				Assim, as tensões após o FAA e o bloco de ganho são:

				\begin{itemize}	
					\item Tensão média de saída: $V_{o(av)} = G\* V_{i(av)}$;
					\item Tensão máxima de saída: $V_{o(max)} = G\* (V_{i(av)} + \frac{V_{i(pp)}}{2})$;
				\end{itemize}	
					

				A excursão do sinal de $V_{o(av)}$ até $V_{o(max)}$ representa a excursão de aceleração
				de $0g$ a $50g$. Assim, a excursão de 1mg representa um sinal de tensão de:

				\begin{gather*}
					V_{o(1mg)} = \frac{V_{o(max)} - V_{o(av)}}{50000mg}\\
					V_{o(1mg)} = \frac{G\*V_{i(av)} + G\*\frac{V_{i(pp)}}{2} - G\*V_{i(av)}}{50000}\\
					V_{o(1mg)} = \frac{G\*V_{i(pp)}}{100000}\\
					V_{o(1mg)} = 10\*G/100000\\
					V_{o(1mg)} = G/10000
				\end{gather*}
				
				E $50mg$ representam $50$ vezes o valor anterior. Logo:
				\begin{gather*}
					V_{o(50mg)} =50\*V_{o(1mg)}\\
					V_{o(50mg)} =50\*G/10000\\
					V_{o(50mg)} =G/200
				\end{gather*}
				Percebe-se que o valor da tensão de saída independe da tensão média
				de entrada, podendo esta variar entre os valores nominais do sensor
				de $10V$ a $14V$. Isso se dá uma vez que realiza-se a correção do
				\textit{offset}, tornando $V_{o(50mg)}$ um valor diferencial. Assim,
				o valor da tensão depende apenas do ganho.
				
				Por sua vez, o ganho é dado por:
				\begin{gather*}
					G = V_{o(max)}/V_{i(max)}
				\end{gather*}
				Em que $V_{o(max)}$ é a tensão de alimentação do conversor ADC e $V_{i(max)}$ é
				$19V$. Da expressão do número de \textit{bits} do ADC, têm-se que:
				\begin{gather*}
					LSB = \frac{V_{o(max)}}{2^n}\\
					ou\\
					V_{o(max)} = LSB\times 2^n
				\end{gather*}
				em que:\\
				\textit{LSB} (\textit{Least Significant Bit}) é a resolução do ADC;\\
				\textit{n} é o número de \textit{bits} do ADC.\\
				
				Como já citado, a menor resolução da grandeza a ser lida deve ser de
				$50mg$. Sendo a sensibilidade do sensor $100mV/g$, então $50mg$
				correspondem a $5mV$ de excursão na tensão de entrada. Dessa forma,
				pode-se escrever LSB em função da mínima resolução desejada.
				\begin{gather*}
					LSB = 5\times 10^{-3}\times G
				\end{gather*}
				Substituindo na equação do ganho:
				\begin{gather*}
					G = V_{o(max)}/V_{i(max)}\\
					G = \frac{LSB\times 2^n}{V_{i(max)}}\\
					G = \frac{G\times 5\times 10^{-3}\times 2^n}{V_{i(max)}}\\
					1 = \frac{5\times 10^{-3}\*2^n}{V_{i(max)}}\\
					2^n = \frac{V_{i(max)}}{5\times 10^{-3}}\\
					n = log_2\left( \frac{V_{i(max)}}{5\times 10^{-3}} \right) = log_2\left( \frac{19}{5\times 10^{-3}}\right)\\
					n = 11,89
				\end{gather*}
				Como o número de \textit{bits} do conversor ADC deve ser um número inteiro, \textit{n}
				deve ser minimamente $12$ \textit{bits} para atender às condições de contorno. É
				interessante pontuar que o número de \textit{bits} independe da tensão de
				alimentação do conversor ADC, seja ela $3,3V$ ou $5V$. E, consequentemente,
				independe do ganho $G$.
				
				Para o cálculo do ganho $G$ admite-se o pior caso, ou seja, um sinal médio
				de saída de $14V$ com excursão positiva máxima, gerando um sinal de $19V$. O
				ganho deve ser tal que $19V$ seja atenuado para $5V$ ou $3,3V$. Para isso, o
				ganho deve ser de, respectivamente, $0,263V/V$ ou $0,1737V/V$.

			\section{Dimensionamento do FAA}

				A banda de rejeição é caracterizada por ter atenuação
				suficiente para que as frequências inseridas naquela banda possuam amplitude
				inferior ao LSB do ADC. Assim, o ganho na banda de rejeição $H(f_{SB})$ é
				dada por:
				\begin{gather*}
					H(f_{SB}) = LSB/V_{o(max)}\\
					H(f_{SB}) = \frac{LSV}{2^n\times LSV}\\
					H(f_{SB}) = 2^{-n}V/V\\
					ou\\
					H(f_{SB})_{[dB]} = -20\*log(2^n) dB
				\end{gather*}
				Como já calculado anteriormente, $n=12$. Assim:
				\begin{gather*}
					H(f_{SB})_{[db]} = -72,25 dB
				\end{gather*}
				Os filtros analógicos, a partir da frequência de corte $f_c$, possuem
				atenuação dada por:
				\begin{gather*}
					H(f) = -m\times 20 dB/dec, para f>f_c
				\end{gather*}
				Em que m é a ordem do filtro. Logo, um filtro de sexta ordem levaria
				menos de uma década para atenuar o sinal em $72,25dB$.

				O ensaio de vibração que baseia o projeto possui, em sua grande
				maioria, excitações de frequência fixa com eventuais picos. O que
				interessa para a análise da estrutura é se a amplitude em
				determinada frequência é maior que a prevista no modelo. Ou se a
				estrutura sofreu algum impacto cuja amplitude ultrapassa determinado
				valor. Então, é importante que o filtro possua uma banda de passagem
				plana, com o mínimo \textit{ripple} possível, para que as amplitudes não
				variem dentro da banda de passagem. Como, em geral, apenas uma
				frequência é presente, não há a necessidade da característica de
				fase do filtro ser linear.

				Com isso, optou-se pelo filtro de topologia
				\textit{Butterworth}, que apresenta banda de passagem plana. O
				filtro ainda é caracterizado por ser de sexta ordem, com
				frequência de corte de $2kHz$. Esta frequência foi escolhida um
				pouco acima da frequência de interesse pelo fato do ganho da
				frequência de corte já ser de $-3dB$. Assim, se a frequência de
				$1kHz$ fosse escolhida como frequência de corte, os sinais com
				essa frequência seriam atenuados em cerca de $30\%$.

				Os amplificadores operacionais utilizados no FAA devem
				possuir \textit{offset} de tensão cumulativo aos quatro CIs
				menor que $LSB$. No pior caso, em que o ADC opera com $3,3V$,
				o \textit{offset} de cada CI (Circuito Integrado) não pode ser
				maior que $201,4\mu V$. Além disso devem permitir alimentação
				não simétrica de até $19V$.O amplificador operacional de apenas
				94 centavos OP07CDR atende à tais especificações com um \textit{offset}
				típico de $60\mu V$.

				Para o projeto do filtro, utilizou-se o \textit{software}
				FilterPro\textsuperscript{TM} da Texas
				Instruments\textsuperscript{TM}. As Figuras ~\ref{fig:circuitoFAA},
				~\ref{fig:faaDiagramaBode} e ~\ref{fig:faaDiagramaBodeZoom} apresentam o
				circuito recomendado para o filtro passa-baixa de 6ª ordem
				\textit{Butterworth} com topologia \textit{Sallen Key} de ganho
				unitário, frequência de corte $f_c = 2kHz$ e frequência de
				rejeição $f_{SB}\simeq 8,2kHz$.

				\begin{figure}[!ht]
					\centering
					\includegraphics[width=\linewidth]{../../Fotos/filterPro.jpg}
					\caption{Filtro anti-\textit{aliasing}}
					\label{fig:circuitoFAA}
				\end{figure}

				\begin{figure}[!ht]
					\centering
					\includegraphics[width=\linewidth]{../../Fotos/filterProGF.jpg}
					\caption{Diagrama de Bode do FFA}
					\label{fig:faaDiagramaBode}
				\end{figure}

				\begin{figure}[!ht]
					\centering
					\includegraphics[width=\linewidth]{../../Fotos/filterProZoom.jpg}
					\caption{\textit{Zoom} na banda de transição}
					\label{fig:faaDiagramaBodeZoom}
				\end{figure}

				Após o FAA, o nível do sinal é atenuado para valores abaixo de
				3,3V por meio do ganho $G$. Como já citado anteriormente, para
				alimentação do ADC igual  a 3,3V, o ganho deve ser igual a $G =
				0,1737V/V$. Essa relação, dada por um divisor de tensão, pode
				ser aproximada pro resistores de $15k\Omega$ e $75k\Omega$, de
				forma que:
				\begin{gather*}
					G = \frac{15k}{15k+75k} = 0,1667V/V
				\end{gather*}

				Como já mostrado, o número de \textit{bits} e o FAA independem do valor
				de $G$, sendo que tal variação irá afetar apenas na calibração dos
				canais. Por tal calibração já fazer-se necessária, também é
				dispensável o uso de resistores de precisão, uma vez que cada
				canal será calibrado individualmente.

				A partir da Figura ~\ref{fig:faaDiagramaBodeZoom}, nota-se que a banda de rejeição inicia em
				$8,2kHz$. Assim, pelo teorema da amostragem seria necessária uma amostragem
				de, pelo menos, $16,4kHz$ para que não ocorra \textit{aliasing}. Contudo,
				é indiferente qualquer \textit{aliasing} que apareça fora da banda de interesse.
				Dessa forma, é passível que ocorra \textit{aliasing} de $f=1kHz$ em diante. Tomando
				uma margem de segurança de aproximadamente $500Hz$, uma frequência de amostragem
				de $f_s=10kHz$ não elimina o \textit{aliasing} na banda de rejeição, mas, sim, na banda
				de interesse.

				A tolerância dos resistores e capacitores do FAA influenciam no
				deslocamento para mais ou para menos das frequências de corte e
				de rejeição no diagrama de magnitude. Porém, como foi dado uma
				margem de $1kHz$ para a frequência de corte e de $2kHz$ para o
				início da banda de rejeição, novamente torna-se dispensável a
				utilização de componentes de precisão.
				\newpage

			\section{Fonte de corrente}

				A fonte de corrente usará a topologia da Figura ~\ref{fig:topologiaFonteCorrente}.

				\begin{figure}[!ht]
					\centering
					\includegraphics[scale = 0.3]{../../Fotos/fonteCorrenteClean2.jpg}
					\caption{Fonte de corrente}
					\label{fig:topologiaFonteCorrente}
				\end{figure}

				Admitindo uma impedância de medição infinita e alpha dos transistores
				$\alpha\simeq1$, a corrente entregue ao sensor \textit{IEPE} será dada por:
				\begin{gather*}
					I_s = V_{be}/R_1
				\end{gather*}
				Assim, considerando $V_{be}$ aproximadamente constante e igual a $0,7V$ e
				$I_s = 4mA$:
				\begin{gather*}
					R_1 = V_{be}/I_s = 170\Omega
				\end{gather*}
				O resistor $R_2$ garante a polarização de ambos transistores. A
				corrente necessária para polarizar o $Q_1$ será aproximadamente a
				corrente total que passa pelo resistor $R_2$. Segundo o
				\textit{datasheet} do transistor PNP BC857, para $V_{be} = 0,7V$
				tem-se $I_c = 2mA$. $R_2$ possui queda de tensão de $19V -
				2\*V_{be}$. Assim:
				\begin{gather*}
					R_2 = \frac{19 - 2\*V_{be}}{0,002} = 8k88\Omega
				\end{gather*}
				Após montar e simular o circuito, utilizando o \textit{software}
				LTSpice, foi possível realizar um ajuste fino para reduzir a
				corrente de polarização do transistor para cerca de $350\mu A$,
				fazendo com que ambos $V_{be}s$ ficassem em torno de $0,6V$. Assim,
				o novo valor de $R_1$ e $R_2$, aproximado para valores comerciais,
				são:
				\begin{gather*}
					R_1 = 0,6/0,004 = 150\Omega\\
					R_2 = \frac{19-2\times 0,6}{0,00035} = 51k\Omega
				\end{gather*}

				\begin{figure}[!ht]
					\centering
					\includegraphics[scale = 0.3]{../../Fotos/fonteCorrente2.jpg}
					\caption{Fonte de corrente com medições}
				\end{figure}

				A fonte de corrente que alimenta o sensor IEPE deve possuir
				corrente de saída de $2mA$ a $10mA$. Como o valor da corrente
				escolhido é um valor intermediário, $4mA$, os resistores da
				fonte de corrente não precisam ter valores precisos. Resistores
				de 5\% podem ser usados sem influenciar nos resultados.

				Os transistores devem ser transistores de sinal, com
				$V_{be}\simeq 0,7V$. Assim, utilizou-se os transistores BC857B
				que são facilmente encontrados e possuem baixo custo.

			\section{Microcontrolador}
				Uma vez definido a taxa de amostragem, a resolução necessária do
				ADC e o número de canais, é possível a escolha do
				microcontrolador e do módulo de telemetria.

				A restrição do ADC de $12$ \textit{bits} e a venda no Brasil
				restringe a escolha de microcontrolares a apenas o
				STM32F103C8T6.\cite{stm} Nenhum outro microcontrolador com conversor de
				$12$ \textit{bits} foi encontrado à venda.

				Apesar disso, o microcontrolador possui características que
				atendem perfeitamente às especificações do projeto. Como
				principais características, pode-se citar:

				\begin{itemize}
					\item Arquitetura de 32 \textit{bits}
					\item $72MHz$ de frequência de \textit{clock} com 1.25 DMIPIS/MHz
					\item 2xADC de $12$ \textit{bits} com até 1MS/s
					\item Tensão de alimentação de $2V$ a $3.6V$
					\item 2xSPI
					\item 3xUSART
					\item 2x$I^2C$
				\end{itemize}

				A capacidade de conversão de até 1MS/s permite que o
				microcontrolador amostre os dados e os transmita para o módulo
				de telemetria em tempo hábil.
				
				O microcontrolador ainda pode ser encontrada em módulos com os
				componentes básicos para utilização como capacitores e barra de
				pinos para programação. Esse módulo é vendido no Brasil por
				cerca de 20 reais.

				Existem diversas interfaces de programação que podem ser
				utilizadas. A IDE (\textit{Integrated Development Environment })
				disponibilizada gratuitamente pelo fabricante é o \textit{System
				Workbench for STM32}, ou \textit{SW4STM32}, será utilizada.

				Para a programação, utiliza-se o programador ST-Link/V2 que
				conecta USB no computador e faz a interface de programação via
				JTAG.

				Segundo simulações utilizando o \textit{software} STM32CubeMX, o
				microcontrolador possui um consumo médio de $30,56mA$, como pode ser
				visto na Figura ~\ref{fig:consumoSTM}.

				\begin{figure}[!ht]
					\centering
					\includegraphics[scale = 0.6]{../../Fotos/stmConsumo.jpg}
					\caption{Consumo estimado do STM32F103C8T6}
					\label{fig:consumoSTM}
				\end{figure}

			\section{Telemetria}
				Como cada dado possui $12$ \textit{bits} e cada placa possui um
				máximo de $4$ canais, é necessário que o módulo de telemetria
				tenha uma taxa de transferência de:

				\begin{gather*}
					f_s = 10kHz
					Total de bytes por amostra = 4\times 12 = 48 bits = 6 bytes
					B/s = 6\times 10000 = 60kB/s
				\end{gather*}

				O módulo nRF24L01+\cite{nrf} possui taxa de comunicação de até $2Mbps$,
				equivalente a $250kBps$. Outras características do módulo são
				citadas abaixo.

				\begin{itemize}
					\item Frequência de operação de $2,4GHz$
					\item Tensão de alimentação de $1,9V$ a $3,6V$
					\item Half-duplex
					\item Recebe dados de até 6 módulos no mesmo canal
					\item Comunicação SPI de até $10Mbps$
					\item Alcance de comunicação de até $1km$ (ideal)
				\end{itemize}

				Assim, esse módulo se torna ideal para que a placa receptora consiga
				obter dados de até 6 placas de 4 canais, possibilitando a leitura de
				um total de 24 acelerômetros.

				\begin{figure}[!ht]
					\centering
					\includegraphics[scale = 0.6]{../../Fotos/nrf.jpg}
					\caption[Módulo nRF24L01+]{Módulo nRF24L01+ \footnotemark}
				\end{figure}

				\footnotetext{Acessado(03/06/2018): https://www.epictinker.com/IT-NRF24L01-PA-LNA-p/it-nrf24l01-pa-lna.htm}

				O nível lógico se adequa ao do microcontrolador bem como a fonte
				de alimentação. Segundo o \textit{datasheet}, o módulo possui
				consumo médio de $40mA$ em transmissão com picos de $115mA$.

			\section{Fonte Linear}
				A fonte de 3,3V deve alimentar o microcontrolador e o módulo de
				telemetria. O consumo total é de menos de 100mA na média. Os
				picos de corrente do módulo de telemetria devem ser supridos por
				capacitores posicionados próximos aos terminais de alimentação
				do módulo.

				Como a tensão de alimentação é próxima da tensão a ser regulada,
				optou-se por utilizar um regulador linear ao invés de um
				\textit{buck}. Um \textit{buck} teria uma eficiência em torno de
				80\% enquanto, para essa situação, a eficiência de um regulador
				linear é de $3,3/5 = 66\%$. Devido ao preço e à baixa
				complexidade, escolheu-se o CI AMS1117-3.3 que possui capacidade
				de corrente de 1A e requer apenas dois capacitores para
				funcionar.

			\section{Fonte chaveada \textit{boost}}
				Os sensores IEPE operam com uma tensão de alimentação de $18V$ a
				$30V$. A fonte que provê tal alimentação deve ser capaz de
				elevar a tensão de entrada, $5V$, para a tensão desejada e
				alimentar as fontes de corrente dos sensores e o FAA.
				
				Para elevar a tensão, utiliza-se uma fonte chaveada tipo
				\textit{boost}. De forma a manter a uma alta eficiência do
				\textit{boost} e não trabalhar com uma tensão muito próxima do
				limite inferior, a tensão de alimentação escolhida foi de 19V.
				A corrente total que a fonte de 19V deve fornecer é dada por:
				\begin{gather*}
					I_{total} = 4\* \left(I_{FAA} + I_{FonteCorrente}\right)
				\end{gather*}
				O \textit{quatro} representa os quatro canais. A corrente do FAA
				é basicamente a corrente consumida pelos amplificadores
				operacionais que, segundo o \textit{datasheet} do OP07CDR, é de
				abaixo de 1mA para tensões de alimentação superiores a
				aproximadamente 11V. Assim, como cada FAA possui quatro OP07CDR,
				o consumo do FAA é de $4\times 1mA =  4mA$.

				As fontes de corrente consomem os 4mA da polarização do sensor
				mais a corrente que passa pelo resistor de polarização $R_2$.
				Assim, cada filtro consome aproximadamente $4,3mA$. Então:
				\begin{gather*}
					I_{total} = 4\* \left(4mA + 4,3mA\right) = 33,2mA.
				\end{gather*}

				Assim, a fonte \textit{boost} deve ser capaz de elevar a tensão
				de 5V para 19V e fornecer uma corrente de 60mA, aplicando uma
				margem de segurança. Devido ao baixo custo e por atender às
				especificações do projeto, optou-se pelo módulo \textit{boost}
				XL6009. Trata-se de uma fonte chaveada elevadora de tensão com
				capacidade de regular a tensão de saída por meio de um
				\textit{trimpot} até 32V e capacidade de corrente de 4A.

			\section{Consumo total}
				O consumo total é dado pelas cargas das fontes de $19V$ e $3,3V$ e
				suas respectivas eficiências. Assumindo que a eficiência do
				módulo XL6009 como sendo de $75\%$ e que toda a corrente que entra
				na fonte linear é entregue à carga, têm-se que:
				\begin{gather*}
					P = \frac{V_{XL}\times I_{XL}}{\eta _{XL}} + \frac{V_{AMS}\times I_{AMS}}{\eta _{AMS}}\\
					P = \frac{19\times 33,2mA}{0,75} + \frac{3,3\times 70,56mA}{0,66}\\
					P = 1,19W
				\end{gather*}
				Em que o subscrito $XL$ refere-se ao conversor \textit{boost} e
				o subscrito $AMS$ refere-se ao conversor linear.

				O \textit{powerbank} que alimenta a placa possui estimados $1760mAh$ de
				energia. Em \textit{potência-hora}, assumindo uma tensão nominal
				de 5V, a energia total é de 8.8Wh. Assim, a placa consegue operar
				por pouco mais 7h com esse \textit{powerbank}.

        \bibliography{../../Bibliografia/bibliografia}

\end{document}